\documentclass[11pt,a4paper]{moderncv}

% moderncv themes
%\moderncvtheme[blue]{casual}                 % optional argument are 'blue' (default), 'orange', 'red', 'green', 'grey' and 'roman' (for roman fonts, instead of sans serif fonts)
\moderncvtheme[blue]{classic}                % idem
\usepackage{xunicode, xltxtra}
\XeTeXlinebreaklocale "zh"
\widowpenalty=10000

%\setmainfont[Mapping=tex-text]{文泉驿正黑}

% character encoding
%\usepackage[utf8]{inputenc}                   % replace by the encoding you are using
\usepackage{CJKutf8}

% adjust the page margins
\usepackage[scale=0.8]{geometry}
\recomputelengths                             % required when changes are made to page layout lengths
%\setmainfont[Mapping=tex-text]{微软雅黑}
\setmainfont[Mapping=tex-text]{文泉驿微米黑}
\setsansfont[Mapping=tex-text]{微软雅黑}
% \setsansfont[Mapping=tex-text]{Hiragino Sans GB}
\CJKtilde

% personal data

%% start of file `template-zh.tex'.
%% Copyright 2006-2012 Xavier Danaux (xdanaux@gmail.com).
%
% This work may be distributed and/or modified under the
% conditions of the LaTeX Project Public License version 1.3c,
% available at http://www.latex-project.org/lppl/.

% 个人信息
\firstname{骆}
\familyname{克云}
\title{个人简历}                      % 可选项、如不需要可删除本行
\address{南京市仙林大道163号}{210046 江苏南京}             % 可选项、如不需要可删除本行
\mobile{+86~18115128082}                         % 可选项、如不需要可删除本行
%\phone{+2~(345)~678~901}                          % 可选项、如不需要可删除本行
%\fax{+3~(456)~789~012}                            % 可选项、如不需要可删除本行
\email{streamer.ky@foxmail.com}                    % 可选项、如不需要可删除本行
%\homepage{keyunluo.github.io}                  % 可选项、如不需要可删除本行
%\extrainfo{附加信息 (可选项)}                  % 可选项、如不需要可删除本行
\photo[64pt]{pic.jpg}                  % ‘64pt’是图片必须压缩至的高度、‘0.4pt‘是图片边框的宽度 (如不需要可调节至0pt)、’picture‘ 是图片文件的名字;可选项、如不需要可删除本行
%\quote{引言(可选项)}                           % 可选项、如不需要可删除本行

% 显示索引号;仅用于在简历中使用了引言
%\makeatletter
%\renewcommand*{\bibliographyitemlabel}{\@biblabel{\arabic{enumiv}}}
%\makeatother

% 分类索引
%\usepackage{multibib}
%\newcites{book,misc}{{Books},{Others}}
%----------------------------------------------------------------------------------
%            内容
%----------------------------------------------------------------------------------
\begin{document}
\maketitle

\section{教育背景}
\cventry{2016 -- 2019}{研究生}{南京大学}{计算机科学与技术系} {分布式计算研究组}{}
\cventry{2012 -- 2016}{本科}{南京航空航天大学}{计算机科学与技术学院}{计算机软件培优班}{}


\section{研究方向}
\cventry{研究方向}{分布计算与并行处理}{分布式计算平台Spark的性能优化}{大数据处理}{}{}
\cvitem{本科毕业设计}{\emph{基于HDFS的大数据存储系统的设计与实现}}
\cvitem{指导老师}{刘亮,叶保留}
\cvitem{说明}{\small 本毕设研究使用Spark编写的BulkLoad模块将HDFS中的数据快速复制到HBase中,同时在Elasticsearch中建立索引同步机制。}

% 来自BibTeX文件但不使用multibib包的出版物
%\renewcommand*{\bibliographyitemlabel}{\@biblabel{\arabic{enumiv}}}% BibTeX的数字标签
\nocite{*}
\bibliographystyle{plain}
\bibliography{publications}                    % 'publications' 是BibTeX文件的文件名

% 来自BibTeX文件并使用multibib包的出版物
\section{论文}
%\nocitebook{book1,book2}
%\bibliographystylebook{plain}
%\bibliographybook{publications}               % 'publications' 是BibTeX文件的文件名
%\nocitemisc{misc1,misc2,misc3}
%\bibliographystylemisc{plain}
%\bibliographymisc{publications}               % 'publications' 是BibTeX文件的文件名

\section{社区}
\cventry{Blog}{\url{https://keyunluo.github.io}}{技术博客}{记录、总结日常学习的知识点}{}{}
\cventry{GitHub}{\url{https://github.com/keyunluo}}{参与和创建过多个开源项目}{}{}{}
%\cventry{StackOverflow}{\url{http://stackoverflow.com/users/1983467/dinever}}{}{}{}{}

\section{项目经历}
\renewcommand{\baselinestretch}{1.2}

\cventry{2016}
{网页批量注册信息}
{Python/Tesseract/curl}
{个人项目}{}
{采用正则表达式/XPath搜索指定信息,使用Pycurl发送信息,使用Tesseract训练并识别验证码。}

\cventry{2015}
{政府采购网爬虫项目}
{Python/Scrapy/PyQt/MongoDB}
{团队项目}{}
{采用Python下的Scrapy爬虫框架编写了一个上海市政府采购网招标信息爬取程序,数据存储在MongoDB中,使用PyQt编写界面,可实现数据采集和简单查询。}

\cventry{2014--2015}
{基于图像识别的邮件清单分拣系统}
{C++/MFC}
{大学生创新项目 团队项目}{}
{作为团队负责人,设计并实现了对快递业邮件清单的图像处理、分析系统,使用Tesseract进行文字识别,使用MFC并进行界面美化。}


\section{语言技能}
\cvline{英语}{\textbf{CET-6: 487},擅长读写,经常阅读英文论文、教程。}
\cvline{普通话}{母语}

\section{计算机技能}
\cvline{编程语言}{Python,Java,Scala,C,C++}
\cvline{数据库}{HBase,MySQL}
\cvline{大数据平台}{Spark,Hadoop,Flink,Kudu}


\section{荣誉} % (fold)
\cvitem{2016}{三好学生}
\cvitem{2015}{校第三届嵌入式比赛二等奖,第二届PLD电子系统设计竞赛二等奖}
\cvitem{2014}{院第三届科创基金项目一等奖,校第九届程序设计比赛二等奖}
\cvitem{2013}{优秀学生二等奖学金,校首届高等数学竞赛二等奖}

\closesection{}                   % needed to renewcommands
\renewcommand{\listitemsymbol}{-} % change the symbol for lists


\end{document}


%% 文件结尾 `template-zh.tex'.
